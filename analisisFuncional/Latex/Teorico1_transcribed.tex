% Transcription of Teórico 1.md into LaTeX
\documentclass[11pt]{article}
\usepackage[utf8]{inputenc}
\usepackage[T1]{fontenc}
\usepackage{amsmath,amssymb,amsthm}
\usepackage{enumitem}
\usepackage{hyperref}
\usepackage{cleveref}

\theoremstyle{definition}
\newtheorem{definition}{Definición}[section]
\newtheorem{example}[definition]{Ejemplo}
\newtheorem{remark}[definition]{Remark}
\theoremstyle{plain}
\newtheorem{lemma}[definition]{Lema}
\newtheorem{proposition}[definition]{Proposición}
\newtheorem{theorem}[definition]{Teorema}
\newtheorem{corollary}[definition]{Corolario}

\begin{document}

% metadata
% dateCreated: 2024-09-21,00:32
% tags: Riesz, FunctionalAnalysis, normEquivalences, banachSpace, vectorSpace

\section*{Espacio normado}

\begin{definition}[Espacio normado]
Sea $X$ un espacio vectorial sobre $\mathbb{F}$ (\(\mathbb{R}\) o \(\mathbb{C}\)). Una norma en $X$ es una función $\|\cdot\|:X\to\mathbb{R}$ tal que para todo $x,y\in X$ y $\alpha\in\mathbb{F}$ se cumple:
\begin{enumerate}[label=(\arabic*)]
\item $\|x\|\ge 0$.
\item $\|x\|=0\iff x=0$.
\item $\|\alpha x\|=|\alpha|\,\|x\|$.
\item $\|x+y\|\le\|x\|+\|y\|$ (desigualdad triangular).
\end{enumerate}
\end{definition}

\section*{Ejemplos de normas}

\begin{example}
Sea $\mathbb{F}^n$. La norma estándar es
$$\|x\|_2=(\sum_{j=1}^n|x_j|^2)^{1/2},$$
que es una norma en $\mathbb{F}^n$.
\end{example}

\begin{example}
Sea $X$ un e.v. de dimensión finita con base $\{e_1,\dots,e_n\}$. Para
$x=\sum\lambda_j e_j$ y $y=\sum\mu_j e_j$ definimos
$$\|x\|=\Big(\sum|\lambda_j|^2\Big)^{1/2}.$$
Se verifica (1)--(3) fácilmente; para (4) se usa que
\begin{align*}
\|x+y\|^2&=\sum|\lambda_j+\mu_j|^2=\sum\big(|\lambda_j|^2+2\mathrm{Re}(\lambda_j\overline{\mu_j})+|\mu_j|^2\big)\\
&\le\sum|\lambda_j|^2+2\sum|\lambda_j||\mu_j|+\sum|\mu_j|^2\\
&\le\|x\|^2+2\|x\|\|y\|+\|y\|^2=(\|x\|+\|y\|)^2,
\end{align*}
por Hölder (caso $p=q=2$).
\end{example}

\begin{example}
Sea $M$ un espacio métrico compacto y $C_{\mathbb{F}}(M)$ el espacio de funciones continuas (omito detalles).
\end{example}

\begin{example}
Sea $(X,\Sigma,\mu)$ un espacio medible y considere $L^p(X)$, $1\le p\le\infty$:
\begin{enumerate}[label=(\alph*)]
\item Si $1\le p<\infty$, $\|f\|_p=(\int|f|^p)^{1/p}$ es una norma.
\item Si $p=\infty$, $\|f\|_\infty=\operatorname{ess\,sup}|f|$ es norma en $L^\infty(X)$.
\end{enumerate}
\end{example}

\begin{example}
Caso particular: $(\mathbb{N},\mathcal{P}(\mathbb{N}),\mu_c)$ con la medida de contar. Toda función $f:\mathbb{N}\to\mathbb{F}$ se identifica con la sucesión $\{a_n\}$, $a_n=f(n)$. Entonces
$f$ integrable respecto a $\mu_c$ es equivalente a $\sum_{n=1}^\infty|a_n|<\infty$ y
$$\int_{\mathbb{N}} f\,d\mu_c=\sum_{n=1}^\infty a_n.$$
Definimos $\ell^p$ como el conjunto de sucesiones $x=(x_n)$ tales que $\sum|x_n|^p<\infty$ ($1\le p<\infty$), y $\ell^\infty$ como el conjunto de sucesiones acotadas. Las normas son
$$\|x\|_p=(\sum|x_n|^p)^{1/p},\qquad\|x\|_\infty=\sup_n|x_n|.$$
Por Hölder, para $1/p+1/q=1$ se tiene
$$\sum_{n=1}^\infty|x_ny_n|\le\|x\|_p\|y\|_q.$$
\]
\end{example}

\begin{example}
Si $X$ es un ev con norma $\|\cdot\|$ y $S\subset X$ subespacio, la restricción de la norma a $S$ es una norma en $S$.
\end{example}

\begin{example}
Si $X,Y$ son espacios normados con normas $\|\cdot\|_1,\|\cdot\|_2$, entonces $Z=X\times Y$ con
$$\|(x,y)\|_Z=\|x\|_1+\|y\|_2$$
es un espacio normado.
\end{example}

\begin{remark}
Un espacio vectorial con una norma se llama espacio vectorial normado. Un vector $x$ con $\|x\|=1$ se dice unitario.
\end{remark}

\begin{lemma}
Todo espacio normado es un espacio métrico: si definimos $d(x,y)=\|x-y\|$ entonces $(X,d)$ es métrico.
\begin{proof}
Verificación directa de axiomas de métrica usando propiedades de la norma.
\end{proof}
\end{lemma}

\begin{remark}
No toda métrica proviene de una norma. Si la métrica es homogénea $d(\alpha x,\alpha y)=|\alpha|d(x,y)$ y es invariante por traslación $d(x,y)=d(x+z,y+z)$, entonces existe una norma con $d(x,y)=\|x-y\|$.
\end{remark}

\begin{theorem}[Continuidad de norma, suma y producto]
Sea $X$ espacio vectorial normado. Si $x_n\to x$, $y_n\to y$ y $\alpha_n\to\alpha$ entonces:
\begin{enumerate}[label=(\arabic*)]
\item $|\,\|x\|-\|y\|\,|\le\|x-y\|$.
\item $\lim_{n\to\infty}\|x_n\|=\|x\|$.
\item $\lim_{n\to\infty}(x_n+y_n)=x+y$.
\item $\lim_{n\to\infty}\alpha_n x_n=\alpha x$.
\end{enumerate}
\end{theorem}

\begin{remark}
En un espacio normado, la norma, la suma y el producto por escalar son funciones continuas.
\end{remark}

\begin{definition}[Equivalencia de normas]
Sean $\|\cdot\|_1,\|\cdot\|_2$ normas en $X$. Decimos que son equivalentes si existen constantes $m,M>0$ tales que
$$m\|x\|_1\le\|x\|_2\le M\|x\|_1\qquad\forall x\in X.$$ 
\end{definition}

\begin{corollary}[Propiedades de normas equivalentes]
Si $\|\cdot\|$ y $\|\cdot\|_1$ son equivalentes con métricas asociadas $d,d_1$, entonces para toda sucesión $x_n$ en $X$:
\begin{enumerate}[label=(\arabic*)]
\item $x_n\to x$ en $(X,d)\iff x_n\to x$ en $(X,d_1)$.
\item $\{x_n\}$ es de Cauchy en $(X,d)\iff$ es de Cauchy en $(X,d_1)$.
\item $(X,d)$ es completo $\iff (X,d_1)$ es completo.
\end{enumerate}
\end{corollary}

\begin{theorem}[Función continua en compacto tiene máximo y mínimo]
Sea $(M,d)$ compacto y $f:M\to\mathbb{F}$ continua. Entonces $f$ es acotada y alcanza su supremo e ínfimo; en particular existen $x,y\in M$ con $f(x)=\sup|f|$ y $f(y)=\inf|f|$.
\end{theorem}

\begin{theorem}[Equivalencia a la norma 1]
Sea $X$ espacio vectorial normado de dimensión finita con norma $\|\cdot\|$. Sea $\{e_j\}_{j=1}^n$ una base y defina
$$\|x\|_1=\Big(\sum|\alpha_j|^2\Big)^{1/2}\quad\text{si }x=\sum\alpha_j e_j.$$
Entonces $\|\cdot\|_1$ y $\|\cdot\|$ son equivalentes.
\end{theorem}
\begin{proof}
Sea $M=\big(\sum_{j=1}^n\|e_j\|^2\big)^{1/2}>0$. Para $x=\sum\alpha_j e_j$ tenemos
$$\|x\|=\Big\|\sum\alpha_j e_j\Big\|\le\sum_j|\alpha_j|\,\|e_j\| \le\Big(\sum_j|\alpha_j|^2\Big)^{1/2}\Big(\sum_j\|e_j\|^2\Big)^{1/2}=M\|x\|_1.$$
Por tanto existe $M>0$ con $\|x\|\le M\|x\|_1$ para todo $x$. Para la cota inferior consideramos la aplicación
$$f:\mathbb{F}^n\to\mathbb{R},\qquad f(\alpha_1,\dots,\alpha_n)=\Big\|\sum_j\alpha_j e_j\Big\|.$$ 
La función $f$ es continua y el conjunto
$$S=\{(\alpha_1,\dots,\alpha_n):\sum_j|\alpha_j|^2=1\}$$
es compacto en $\mathbb{F}^n$. Por compacidad $f$ alcanza un mínimo positivo $m>0$ en $S$ (no puede ser cero porque la familia $\{e_j\}$ es base). Si $\|x\|_1=1$ entonces $(\alpha_1,\dots,\alpha_n)\in S$ y por tanto $m\le f(\alpha_1,\dots,\alpha_n)=\|x\|$. Para un $x$ arbitrario tomamos $x'=x/\|x\|_1$ y obtenemos
$$m\|x\|_1 = m\|x'\|_1 \le \|x'\| = \frac{\|x\|}{\|x\|_1},$$
de donde existe $m>0$ con $m\|x\|_1\le\|x\|$ para todo $x$. Esto muestra la equivalencia de normas.
\end{proof}

\begin{corollary}
En dimensión finita todas las normas son equivalentes.
\end{corollary}
\begin{proof}
Sea $\|\cdot\|$ una norma cualquiera en $X$. Por el teorema anterior $\|\cdot\|$ es equivalente a la norma euclidiana $\|\cdot\|_1$ asociada a cualquier base; de ello se sigue por transitividad que cualquier par de normas en $X$ son equivalentes.
\end{proof}

\begin{remark}[Contraejemplo II]
Esto no vale en dimensión infinita: por ejemplo en $C^1[0,\pi]$ las normas $\|\cdot\|_\infty$ y $\|u\|=\|u\|_\infty+\|u'\|_\infty$ no son equivalentes.
\end{remark}

\begin{lemma}
Si $X$ es de dimensión finita y $\|\cdot\|_1$ la norma euclidiana asociada a una base, entonces $(X,d_1)$ es completo (Banach).
\end{lemma}
\begin{proof}
Sea $\{x^n\}$ sucesión de Cauchy. Escribir cada $x^n=\sum_{j=1}^N\alpha_j^n e_j$. Las coordenadas forman sucesiones de Cauchy en $\mathbb{F}$, por completitud convergen a límites $\alpha_j$, y entonces $x=\sum\alpha_j e_j$ es límite en la norma $\|\cdot\|_1$.
\end{proof}

\begin{corollary}
Todo espacio vectorial de dimensión finita es completo para cualquier norma.
\end{corollary}
\begin{proof}
Por el teorema anterior todas las normas en $X$ son equivalentes, luego la completitud es independiente de la norma: si una norma produce un espacio completo entonces todas lo hacen. Como $\|\cdot\|_1$ (la norma euclidiana) hace a $X$ completo (producto de campos completos), se concluye que $X$ es completo para cualquier norma.
\end{proof}

\begin{theorem}[Resultados en métricos]
Sea $(M,d)$ métrico y $A\subset M$. Entonces:
\begin{enumerate}[label=(\arabic*)]
\item Si $A$ es completo entonces $A$ es cerrado.
\item Si $M$ es completo, $A$ es completo $\iff A$ es cerrado.
\item Si $A$ es compacto entonces $A$ es cerrado y acotado.
\item En $\mathbb{F}^n$, cerrado y acotado $\Rightarrow$ compacto.
\end{enumerate}
\end{theorem}

\begin{corollary}
Todo subespacio vectorial de dimensión finita es cerrado.
\end{corollary}

\begin{remark}[Contraejemplo I]
No es cierto en dimensión infinita: hay subespacios de dimensión infinita que no son cerrados (ejemplos en $\ell^\infty$).
\end{remark}

\begin{lemma}
La clausura de un subespacio es subespacio.
\end{lemma}
\begin{proof}
Si $x_n\to x$ y $y_n\to y$ con $x_n,y_n\in S$, entonces $x_n+y_n\to x+y$ y $\alpha x_n\to\alpha x$, por lo que $x+y,\alpha x\in\overline S$.
\end{proof}

\begin{definition}[Span]
Para $E\subset X$ definimos
$$\operatorname{Sp}(E)=\{\text{combinaciones lineales finitas de elementos de }E\},\qquad\overline{\operatorname{Sp}}(E)=\bigcap\{M: M\text{ subespacio cerrado que contiene }E\}.$$ 
\end{definition}

\begin{lemma}
Se cumple $\overline{\operatorname{Sp}}(E)=\overline{\operatorname{Sp}(E)}$ y otras propiedades estándar.
\end{lemma}

\begin{lemma}[Lema de Riesz]
Sea $X$ normado y $Y\subset X$ subespacio cerrado, $Y\ne X$. Para $\alpha\in(0,1)$ existe $x_\alpha\in X$ con $\|x_\alpha\|=1$ y $\|x_\alpha-y\|>\alpha$ para todo $y\in Y$.
\end{lemma}
\begin{proof}
Tomar $x\in X\setminus Y$, definir $d=\inf_{z\in Y}\|x-z\|>0$ y elegir $z\in Y$ con $d<\|x-z\|<d/\alpha$. Poner $x_\alpha=(x-z)/\|x-z\|$.
\end{proof}

\begin{theorem}
Si $X$ es de dimensión infinita, los conjuntos
$$D=\{x:\|x\|\le1\},\qquad K=\{x:\|x\|=1\}$$
no son compactos.
\end{theorem}
\begin{proof}
Construir una sucesión en $K$ sin sub-sucesión convergente usando el lema de Riesz iterativamente; produce puntos mutuamente separados por al menos $1/2$.
\end{proof}

\begin{definition}[Espacio de Banach]
Un espacio de Banach es un espacio normado completo.
\end{definition}

\begin{theorem}
\begin{enumerate}[label=(\arabic*)]
\item Todo espacio normado de dimensión finita es Banach.
\item Si $X$ es métrico completo, $C_\mathbb{F}(X)$ es Banach.
\item Si $(X,\Sigma,\mu)$ es espacio de medida, $L^p(X)$ ($1\le p\le\infty$) son Banach.
\item Si $X$ es Banach y $Y$ subespacio, entonces $Y$ es Banach $\iff Y$ es cerrado.
\end{enumerate}
\end{theorem}

\begin{theorem}
Sea $X$ Banach y $\{x_n\}\subset X$. Si $\sum\|x_n\|$ converge, entonces $\sum x_n$ converge (absoluta implica convergencia en Banach).
\end{theorem}
\begin{proof}
Las sumas parciales forman una sucesión de Cauchy: $\|\sum_{k=n+1}^m x_k\|\le\sum_{k=n+1}^m\|x_k\|$, y como la serie de normas converge, las sumas parciales son Cauchy; por completitud convergen.
\end{proof}

\end{document}
