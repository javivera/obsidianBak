\documentclass{article}
\usepackage{amsmath, amssymb, amsthm}
\usepackage{enumitem}
\usepackage{hyperref}
\begin{document}

% --- Producto Interno ---

\section*{Producto Interno}

\textbf{Definición:} Un espacio vectorial sobre $\mathbb{R}$ ($\mathbb{C}$). Un producto interno en $X$ es una función $X \times X \to \mathbb{R}$ ($\mathbb{C}$) tal que para todo $x, y, z \in X$, $\alpha, \beta \in \mathbb{R}$ ($\mathbb{C}$):
\begin{itemize}
    \item[a)] $\langle x, x \rangle \ge 0$ (y $\langle x, x \rangle \in \mathbb{R}$)
    \item[b)] $\langle x, x \rangle = 0 \iff x = 0$
    \item[c)] $\langle \alpha x + \beta y, z \rangle = \alpha \langle x, z \rangle + \beta \langle y, z \rangle$
    \item[d)] $\langle x, y \rangle = \langle y, x \rangle$ \quad ($\langle x, y \rangle = \overline{\langle y, x \rangle}$)
\end{itemize}

\textbf{Ejemplos:}
\begin{itemize}
    \item $(\cdot, \cdot): \mathbb{C}^n \times \mathbb{C}^n \to \mathbb{C}$ dado por $\langle x, y \rangle = \sum x_k \bar{y}_k$
    \item $\langle \cdot, \cdot \rangle: \mathbb{R}^n \times \mathbb{R}^n \to \mathbb{R}$ dado por $\langle x, y \rangle = \sum_{k=1}^n x_k y_k$
    \item Si $X$ es un ev con base $\{e_1, \dots, e_k\}$ y $x = \sum \lambda_k e_k$, $y = \sum \beta_k e_k$, entonces $\langle x, y \rangle = \sum \lambda_k \bar{\beta}_k$
    \item Si $(X, \Sigma, \mu)$ es medible, entonces $\langle f, g \rangle = \int f \overline{g}$ es producto interno en $L^2(X)$
    \item Para $a = \{a_n\} \in \ell^2$, $b = \{b_n\} \in \ell^2$, entonces $\langle a, b \rangle = \sum a_n \overline{b_n}$
    \item Si $X$ ev con $\langle \cdot, \cdot \rangle$ y $S$ subespacio, la restricción es producto interno en $S$
    \item Si $X, Y$ son ev con productos internos $\langle \cdot, \cdot \rangle_1$, $\langle \cdot, \cdot \rangle_2$, entonces en $Z = X \times Y$:
    \[
    \langle (u, v), (x, y) \rangle_Z = \langle u, x \rangle_1 + \langle v, y \rangle_2
    \]
\end{itemize}

\textbf{Lema:} (Resultados evpi) Para $x, y, z \in X$, $\alpha, \beta \in \mathbb{F}$:
\begin{itemize}
    \item[a)] $\langle 0, y \rangle = \langle x, 0 \rangle = 0$
    \item[b)] $\langle x, \alpha y + \beta z \rangle = \bar{\alpha} \langle x, y \rangle + \bar{\beta} \langle x, z \rangle$
    \item[c)]
    $\langle \alpha x + \beta y, \alpha x + \beta y \rangle = |\alpha|^2 \langle x, x \rangle + |\beta|^2 \langle y, y \rangle + \alpha \bar{\beta} \langle x, y \rangle + \beta \bar{\alpha} \langle y, x \rangle$
\end{itemize}

\textbf{Proposición: Desigualdad de Cauchy-Schwarz}
\begin{itemize}
    \item[a)] $|\langle x, y \rangle|^2 \le \langle x, x \rangle \langle y, y \rangle$
    \item[b)] La función $\|x\| = \langle x, x \rangle^{1/2}$ es norma en $X$
\end{itemize}

\textbf{Norma Inducida:}
\[
\|x\| = \langle x, x \rangle^{1/2}
\]
\[
|\langle x, y \rangle| \le \|x\| \|y\|
\]

% ...continúa con el resto del contenido siguiendo el mismo formato...

\end{document}
