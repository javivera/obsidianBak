\documentclass[12pt,a4paper]{article}
\usepackage[utf8]{inputenc}
\usepackage[spanish]{babel}
\usepackage{amsmath,amssymb,amsthm}
\usepackage{geometry}
\usepackage{enumitem}

\geometry{margin=2.5cm}

% Theorem environments
\newtheorem{theorem}{Teorema}[section]
\newtheorem{lemma}[theorem]{Lema}
\newtheorem{proposition}[theorem]{Proposición}
\newtheorem{corollary}[theorem]{Corolario}
\theoremstyle{definition}
\newtheorem{definition}[theorem]{Definición}
\newtheorem{example}[theorem]{Ejemplo}
\theoremstyle{remark}
\newtheorem{remark}[theorem]{Observación}

\title{Análisis Funcional - Teóricos 1 y 2}
\author{}
\date{}

\begin{document}

\maketitle

\section{Teórico 1: Espacios Normados y Espacios de Banach}

\begin{lemma}[Todo normado es métrico]
Sea $X$ espacio vectorial con norma $\|\cdot\|$. Sea $d:X\times X\rightarrow \mathbb{R}$ por $d(x,y)=\|x-y\|$ entonces
$$(X,d)\text{ es métrico}$$
Equivalentemente todo $X$ espacio vectorial normado es métrico con la métrica estándar.
\end{lemma}

\begin{theorem}
Sea $X$ espacio vectorial con norma $\|\cdot\|$. Sea $x_n\rightarrow x$, $y_n\rightarrow y$ con $\{\alpha_n\}\subseteq \mathbb{F}$ y $\alpha_n\rightarrow\alpha$ entonces:
\begin{enumerate}
    \item $\big|\|x\| -\|y\|\big|\leq \|x-y\|$
    \item $\lim_{n \to \infty}\|x_n\|=\|x\|$
    \item $\lim_{n \to \infty}x_n+y_n=x+y$
    \item $\lim_{n \to \infty}\alpha_n x_n=\alpha x$
\end{enumerate}
\begin{proof}
Sale todo usando 1. y 1. sale usando $\|x\|\leq \|x-y\|+\|y\|$.
\end{proof}
\end{theorem}

\begin{remark}
Esto nos dice que en un espacio vectorial normado, la norma, la suma y producto por escalar son continuas.
\end{remark}

\begin{definition}[Equivalencia de normas]
Sea $X$ espacio vectorial y $\|\cdot\|_1,\|\cdot\|_2$ normas en $X$ decimos que son equivalentes si 
$$\exists\, m,M>0\quad /\quad m\|x\|_1\leq \|x\|_2\leq M\|x\|_1$$
\end{definition}

\begin{remark}
La equivalencia de normas es una relación de equivalencia.
\end{remark}

\begin{lemma}
Let $X$ be a vector space with $\|\cdot\|, \|\cdot\|_1$ and $d, d_1$ associated metrics. Suppose $\exists k>0$ such that $\|x\| \leq k \|x\|_1\quad \forall x \in X$. Let $\{x_n\}\subseteq X$, then:
\begin{enumerate}
    \item $x_n \rightarrow x$ in $(X, d_1) \Longrightarrow x_n \rightarrow x$ in $(X, d)$
    \item $\{x_n\}$ is Cauchy in $(X, d_1) \Longrightarrow \{x_n\}$ is Cauchy in $(X, d)$
\end{enumerate}
\begin{proof}
Trivial using the inequality.
\end{proof}
\end{lemma}

\begin{corollary}
Sea $X$ espacio vectorial $\|\cdot\|$ y $\|\cdot\|_1$ normas equivalentes en $X$ con $d,d_1$ métricas asociadas. Sea $\{x_n\}\subseteq X$ entonces:
\begin{enumerate}
    \item $x_n\rightarrow x$ en $(X,d)\iff x_n\rightarrow x$ en $(X,d_1)$
    \item $\{x_n\}$ es de Cauchy en $(X,d)\iff \{x_n\}$ es de Cauchy en $(X,d_1)$
    \item $(X,d)$ es completo $\iff (X,d_1)$ es completo
\end{enumerate}
\begin{proof}
Usando el lema anterior y que equivalencia de normas.
\end{proof}
\end{corollary}

\begin{theorem}[Continua en compacto tiene máximo y mínimo]
Sea $(X,d)$ métrico compacto y $f:M\rightarrow \mathbb{F}$ continua entonces $\exists c>0 \,/\, |f(x)|\quad\forall x \in M$ ($f$ acotada).
En particular si $\mathbb{F}=\mathbb{R}$ los números
$$a = \sup\{|f(x)| :x \in M\}$$
$$b = \inf\{|f(x)| :x \in M\}$$
Existen y son finitos. Más aun 
$$\exists\, x,y \in M \quad f(x)=a \quad \land\quad f(y)=b$$
\end{theorem}

\begin{theorem}
Sea $X$ espacio vectorial normado de dim finita con norma $\|\cdot\|$. Sea $\{e_1,\ldots,e_n\}$ base para $x=\sum\alpha_j e_j$ y sea $\|x\|_1=\left(\sum|\alpha_j|^2\right)^{\frac{1}{2}}$ entonces $\|\cdot\|_1$ y $\|\cdot\|$ son equivalentes.
\begin{proof}
\begin{enumerate}
    \item $M=(\sum \|e_j\|^2)^{\frac{1}{2}}>0$
    \item $\|x\|=\left\|\sum\alpha_j e_j\right\|\leq \sum \|\alpha_j e_j\|=\sum|\alpha_j|\|e_j\| \leq (\sum|\alpha_j|^2)^{\frac{1}{2}}(\sum\|e_j\|^2)^{\frac{1}{2}} = \|x\|_1 M$
    \item $f:\mathbb{F}^n\rightarrow \mathbb{R}\quad f(\alpha_1,\ldots,\alpha_n)=\left\|\sum\alpha_j e_j\right\| =\|x\|$
    \item Ver que es continua
    \item $S=\left\{(\alpha_1,\dots,\alpha_n):\sum|\alpha_j|^2 = 1\right\}$ es compacto
    \item Existe $m = f(u_1,\dots,u_n)$ mínimo
    \item $m >0$ porque $\{e_j\}$ es base
    \item Si $\|x\|_1=1 \iff \sum|\alpha_j|^2=1 \Rightarrow(\alpha_1,\dots,\alpha_n)\in S$
    \item $m\|x\|_1 = m\leq f(\alpha_1,\dots,\alpha_n)=\|x\|$ por ser $m$ mínimo
    \item Si no $\left\|\frac{x}{\|x\|_1}\right\|=1$ luego $m \leq \left\|\frac{x}{\|x\|_1}\right\|$ por el caso de arriba
\end{enumerate}
\end{proof}
\end{theorem}

\begin{corollary}
En dimensión finita todas las normas son equivalentes.
\begin{proof}
Equivalencia de normas es relación de equivalencia entonces es transitiva.
\end{proof}
\end{corollary}

\begin{remark}[Contraejemplo II]
Esto no vale en dimensión infinita $X=C^1[0,\pi]$ tenemos dos normas no equivalentes 
$$\|\cdot\|_{\infty} \quad \text{ y }\quad \|u\| =\|u\|_{\infty}+\|u'\|_{\infty}$$
Considerando la función $u_n(x)=\sin(nx)$ se puede ver fácilmente.
\end{remark}

\begin{lemma}
$X$ espacio vectorial de dim finita y $\|\cdot\|_1 =\left(\sum |\alpha_j|^2\right)^{\frac{1}{2}}$ y $d_1$ la métrica asociada, entonces $(X,d_1)$ es completo (Banach).
\begin{proof}
\begin{enumerate}
    \item Ya sabemos que es métrico (dim finita $N$)
    \item $\{x^n\}\subseteq X$ suc de Cauchy
    \item $x^n\in X,\quad x^n=\sum_{j}^{N}\alpha_j^n e_j\quad \alpha_j^n\in \mathbb{F}$
    \item $\exists m_0\in \mathbb{N}$ tq $\sum |\alpha_j^k-\alpha_j^m|^2=\|x^k-x^m\|_1^2\leq \epsilon^2$
    \item Fijando $j\in \mathbb{N}$ tenemos $|\alpha_j^k-\alpha_j^m|\leq\epsilon^2\quad \forall k,m\geq m_0$
    \item $\{\alpha_j^n\}$ es de Cauchy en un completo tiene límite $\alpha_j$ (vale para cada $j$)
    \item $\exists n_j\in \mathbb{N}$ tq $|\alpha_j^n -\alpha_j| < \frac{\epsilon^2}{N} \quad\forall n\geq n_j$
    \item $\tilde{n}=\max\{n_0,\dots,n_N\}$ y $x=\sum^N_{j=1}\alpha_j e_j$ ($x\in X$ por ser combinación lineal de elementos de la base)
    \item para $m\geq \tilde{n}$ sucede $\|x^n-x\|_1^2=\sum_j^N|\alpha_j^n-\alpha_j|^2\leq \epsilon^2$
    \item $\{x^n\}$ converge por lo tanto $X$ es de completo
\end{enumerate}
\end{proof}
\end{lemma}

\begin{corollary}
Todo espacio vectorial de dim finita es completo con la métrica asociada a cualquier norma.
\begin{proof}
$(X,d)$ completa $\iff(X,d_1)$ completa por equivalencia de normas y $d_1$ completa.
\end{proof}
\end{corollary}

\begin{theorem}
Sea $(M,d)$ métrico y $A\subseteq M$ entonces:
\begin{enumerate}
    \item $A$ completo $\Rightarrow A$ cerrado
    \item $M$ completo $\Rightarrow$ ($A$ completo $\iff A$ cerrado)
    \item Si $A$ es compacto $\Rightarrow$ $A$ es cerrado y acotado
    \item Cerrado y acotado en $\mathbb{F}^n \Rightarrow$ compacto
\end{enumerate}
\end{theorem}

\begin{corollary}
Si $Y$ subespacio vectorial de dim finita $\Rightarrow$ $Y$ es cerrado.
\begin{proof}
\begin{enumerate}
    \item Por ser $Y$ espacio vectorial es normado (norma estándar) por ser espacio vectorial normado es métrico
    \item Por dim finita es completo
    \item Cerrado por el teorema anterior
\end{enumerate}
\end{proof}
\end{corollary}

\begin{remark}[Contraejemplo I]
El corolario anterior no es cierto si la dimensión es infinita.
\begin{proof}
\begin{enumerate}
    \item $S=\bigg\{\{x_n\} \subseteq\ell^{\infty}:\exists n_0\in \mathbb{N} \,/\, x_n=0 \,\forall n\leq n_0\bigg\}$
    \item $S$ subespacio de dim infinita de $\ell^{\infty}$
    \item $x_n = (1, \frac{1}{2},\dots, \frac{1}{n},0,\dots,0,\ldots) \in S$
    \item Sea $x=\left(1, \frac{1}{2},\dots, \frac{1}{n}, \frac{1}{n+1},\dots\right)\in (\ell^{\infty}\setminus S)$
    \item $\|x-x_n\|_{\infty} = \frac{1}{n+1}\rightarrow 0$ o lo mismo $\lim_{n \to \infty}x_n=x$
    \item $S$ no es cerrado
\end{enumerate}
\end{proof}
\end{remark}

\begin{lemma}
$X$ espacio vectorial normado, $S$ subespacio de $X$ entonces $\overline{S}$ es subespacio vectorial de $X$.
\begin{proof}
\begin{enumerate}
    \item $x,y\in \overline{S},\, \alpha \in \mathbb{F}$
    \item Por ser clausura existen $x_n\rightarrow x,\, y_n\rightarrow y$ en $S$
    \item $S$ subesp $x_n+y_n\in S$
    \item $x_n+y_n\rightarrow x+y$ entonces $x+y\in\overline{S}$
    \item Análogo $\alpha x_n$
\end{enumerate}
\end{proof}
\end{lemma}

\begin{definition}[Span]
$E\subseteq X$ normado:
$$\text{Sp}(E) = \{\text{Todas las combinaciones lineales finitas de elementos de } E\}$$
$$\overline{\text{Sp}}(E)=\{\text{Todas las intersecciones de subespacios cerrados que contienen a }E\}$$
\end{definition}

\begin{lemma}
$X$ espacio vectorial normado $\emptyset\neq E \subseteq X$ entonces:
\begin{enumerate}
    \item $\overline{\text{Sp}}(E)$ es un cerrado de $X$ que contiene a $E$
    \item $\overline{\text{Sp}}(E) = \overline{\text{Sp}(E)}$
\end{enumerate}
\begin{proof}
\begin{enumerate}
    \item Intersección de cerrados es cerrados y intersección de subespacios es subespacio
    \item $\overline{\text{Sp}(E)}$ es subespacio cerrado y contiene a $E$ por definición
    \begin{itemize}
        \item ($\subseteq$) $\overline{\text{Sp}}(E)$ es subespacio que contiene a $E$ entonces está en la intersección $\text{Sp}(E)\subseteq\overline{\text{Sp}}(E)$
        \item ($\supseteq$) Como $\overline{\text{Sp}}(E)$ es cerrado y contiene a $\text{Sp}(E)$ luego $\overline{\text{Sp}}(E)\supseteq \overline{\text{Sp}(E)}$
    \end{itemize}
\end{enumerate}
\end{proof}
\end{lemma}

\begin{lemma}[Riesz]
Sea $X$ normado, $Y$ subespacio cerrado con $Y\neq X$. Sea $\alpha \in(0,1)$. Entonces $\exists x_{\alpha}\in X$ con $\|x_{\alpha}\|=1$ tal que $\|x_{\alpha}-y\| >\alpha \quad \forall y\in Y$.
\begin{proof}
\begin{enumerate}
    \item $d=\inf\{\|x-z\|:z\in Y\}>0$
    \item $0<\alpha<1 \Rightarrow d<d\alpha^{-1}$
    \item (Def ínfimo) $\|x-z\|<d\alpha^{-1}$
    \item $x_{\alpha}=\frac{x-z}{\|x-z\|}\quad \|x_{\alpha}\|=1$
    \item $\|x_{\alpha}-y\| =\left\|\frac{x-z}{\|x-z\|}-y\right\| = \frac{1}{\|x-z\|}\|x-(z+\|x-z\|y)\| > \frac{d}{d\alpha^{-1}}$
\end{enumerate}
\end{proof}
\end{lemma}

\begin{theorem}
Sea $X$ espacio vectorial dimensión infinita 
$$D=\{x \in X:\|x\| \leq 1\}\quad K=\{x \in X: \|x\| =1\}$$
no son compactos.
\begin{proof}
\begin{enumerate}
    \item $x_1\in K$ como dim infinita $\text{Sp}(\{x_1\})\neq X$ (Sp cerrado por ser de dim finita)
    \item Por lema de Riesz $\exists x_2\in K$ tal que $\|x_1-x_2\|\geq \frac{1}{2}$
    \item Generalizando $\exists x_n\in K$ tal que $\|x_n-x_m\|\geq \frac{1}{2}\quad\forall n\neq m$
    \item Así armamos una sucesión que no puede tener sub convergente porque si fuese convergente sería de Cauchy
\end{enumerate}
\end{proof}
\end{theorem}

\begin{remark}
$D$ y $K$ son compactos $\Rightarrow$ $X$ dim finita.
\end{remark}

\begin{definition}[Espacio de Banach]
Un espacio de Banach es un espacio normado que es completo con la métrica asociada a la norma.
\end{definition}

\begin{theorem}
\begin{enumerate}
    \item Todo normado de dim finita es Banach
    \item Si $X$ es métrico completo $C_{\mathbb{F}}$ es Banach
    \item Si $(X,\Sigma,\mu)$ subespacio medible $\Rightarrow L^p \,(1\leq p\leq \infty)$ son Banach. (En particular $\ell^p \,(1\leq p\leq \infty)$ son Banach)
    \item $X$ Banach, $Y$ subespacio entonces $Y$ Banach $\iff Y$ cerrado
\end{enumerate}
\begin{proof}
\begin{enumerate}
    \item Visto arriba
    \item Se asume visto en Reales
    \item Se asumen visto en Reales
    \item Por teorema anterior
\end{enumerate}
\end{proof}
\end{theorem}

\begin{theorem}
Sea $X$ de Banach $\{x_n\} \subseteq X$ si la serie $\sum\|x_n\|$ converge entonces $\sum x_k$ converge.
\begin{proof}
\begin{enumerate}
    \item $\left|\sum^m \|x_k\| - \sum^n \|x_k\|\right| =\sum^m_{k=n+1} \|x_k\| \leq\epsilon \quad\forall m\geq n\geq n_0$ (por ser convergente la serie las sumas parciales son de Cauchy)
    \item Sea $S_n=\sum^n x_k$ entonces $\|S_m-S_n\|\leq \left\|\sum^m_{k=n+1}x_k\right\|\leq \sum^m_{k=n+1}\|x_k\|\leq\epsilon$
    \item Como $X$ completo $S_n$ converge
\end{enumerate}
\end{proof}
\end{theorem}

\section{Teórico 2: Espacios con Producto Interno y Espacios de Hilbert}

\begin{proposition}
$X$ e.v.pi $x,y\in X$ entonces:
\begin{enumerate}
    \item $|(x,y)|\leq(x,x)(y,y)$
    \item La función $\|\cdot\| X\rightarrow \mathbb{R}$ dada por $\|x\|=(x,x)^{\frac{1}{2}}$ es norma en $X$
\end{enumerate}
\end{proposition}

\begin{theorem}[Regla Paralelogramo e identidades de polarización]
Sea $X$ e.v.pi con norma inducida $\|\cdot\|$ entonces $\forall x,y\in X$ vale:
\begin{enumerate}
    \item $\|x+y\|^2 + \|x-y\|^2 = 2(\|x\|^2+\|y\|^2)$ (Regla Paralelogramo)
    \item Si $\mathbb{F}=\mathbb{R}\quad\quad 4(x,y)=\|x+y\|^2-\|x-y\|^2$
    \item Si $\mathbb{F}=\mathbb{C}\quad\quad 4(x,y)=\|x+y\|^2-\|x-y\|^2+i\|x+iy\|^2-i\|x-iy\|^2$
\end{enumerate}
\end{theorem}

\begin{proposition}[Continuidad del producto interno]
Sea $X$ e.v.pi. $\{x_n\},\{y_n\}\subseteq X$ con $x_n\rightarrow x,\quad y_n\rightarrow y$ en $X$ entonces
$$(x_n,y_n)\rightarrow(x,y)$$
\end{proposition}

\subsection{Ortogonalidad}

\begin{definition}[Espacio de Hilbert]
Un espacio con producto interno completo con respecto a la métrica asociada a la norma inducida por el producto interno se dice espacio de Hilbert.
\end{definition}

\begin{example}[Contraejemplo I]
$A =\{\{x_n\} \,/\, x_n\neq 0 \text{ solo en finitos n}\}$ es fácil ver que $(\{x_n\},\{y_n\})=\sum^{\infty}_{n=1}x_n\overline{y_n}$ es producto interno con norma inducida $\|\{x_n\}\|=\big(\sum^{\infty}_{n=1}|x_n|^2\big)^{\frac{1}{2}}$. Restaría ver que no es completo.
\end{example}

\begin{proposition}[Subespacio cerrado es Hilbert]
Sea $\mathcal{H}$ Hilbert, $Y\subset H$ sub espacio entonces 
$$Y \text{ Hilbert}\iff Y \text{ es cerrado en }\mathcal{H}$$
\end{proposition}

\begin{lemma}
$X$ e.v.pi $A\subseteq X$ subconjunto entonces:
\begin{enumerate}
    \item $0\in A^{\perp}$
    \item Si $0\in A \in A\Rightarrow A \cap A^{\perp} = \{0\}$ si no $A\cap A^{\perp} = \emptyset$
    \item $\{0\}^{\perp}=X \quad\land\quad X^{\perp}=\{0\}$
    \item Si $A$ contiene una bola $B_a(r)$ para algún $r>0$ y $a\in A\Rightarrow A^{\perp}=\{0\}$. (Si $A$ abierto no vacío $\Rightarrow A^{\perp}=\{0\}$)
    \item $B\subseteq A\Rightarrow A^{\perp}\subseteq B^{\perp}$
    \item $A^{\perp}$ es una sub cerrado de $X$
    \item $A\subseteq (A^{\perp})^{\perp}=A^{\perp\perp}$
\end{enumerate}
\end{lemma}

\begin{proposition}[$x\in Y^{\perp} \iff \|x-y\|\geq \|x\|$]
Sea $Y$ subespacio de $X$ e.v.pi entonces $x\in Y^{\perp} \iff \|x-y\|\geq \|x\|$.
\end{proposition}

\end{document}